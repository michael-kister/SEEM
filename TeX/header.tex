
% make text 4.16'' x 7''
% fleqn, leqno

\documentclass[twoside,11pt]{article}

\usepackage[utf8]{inputenc}

\usepackage[%
%paperwidth=5.16in,
paperwidth=5.5in,
paperheight=8.5in,
left=.5in,
right=.5in,
bottom=.5in,
top=1in % typically =1in, but changing temporarily...
]{geometry}
%

%\renewcommand{\numberline}[1]{#1~}



% we want to indent first paragraphs
\usepackage{indentfirst}

% typography tools
\linespread{.9}
\usepackage{microtype}

% set paragraph indentation
\setlength{\parindent}{2em}


% for section numbering
\usepackage[]{titlesec} 
\newcommand{\periodafter}[1]{#1.}

%\titleformat{\chapter}%
%{\centering\normalsize\sc}
%{\thechapter\periodafter}{1em}{}

\titleformat{\section}%
{\centering\normalsize\sc}
{\thesection\periodafter}{1em}{}

\titleformat{\subsection}%
{\normalsize\itshape}
{\thesection.\thesubsection\periodafter}{1em}{}

\titleformat{\subsubsection}%
{\normalsize\itshape}
{\thesection.\thesubsection.\thesubsubsection\periodafter}{1em}{}


%\renewcommand{\thechapter}{\Roman{chapter}}
\renewcommand{\thesection}{\Roman{section}}
\renewcommand{\thesubsection}{\Alph{subsection}}
\renewcommand{\thesubsubsection}{\roman{subsubsection}}






%----------------------------------
% NEW CENTURY SCHOOLBOOK (QJE)
\usepackage{fouriernc}
\usepackage[cal=cm]{mathalpha} % I prefer these mathcal letters




%=====================================================================
% Stuff for getting the content how you want it


%\usepackage[T1]{fontenc} % hyphenation for accented characters?
\usepackage{mathtools}
\usepackage{enumitem} %\begin{enumerate}[label=Case \arabic*., align=left, leftmargin=*]
\usepackage{amsthm}
   \newtheorem{theorem}{Theorem}
   \newtheorem{prop}{Proposition}
   \newtheorem{lemma}{Lemma}
   \newtheorem{coro}{Corollary}
\usepackage{commath} % calculus: \dif (also has mixed partial derivative...)
\usepackage{tcolorbox} % inset boxes
\usepackage{physics} % for \dv and \pdv
   \newcommand{\wpdv}[2]{\partial #1 / \partial #2}
\usepackage{natbib}
   \bibliographystyle{authordate1}

\usepackage{centernot} % for \centernot\implies
\newcommand{\nimplies}{\centernot\implies}

\usepackage{float}
\usepackage{graphicx}
\usepackage{subcaption}
%\usepackage{subfig} % could be useful for some journals

\usepackage{booktabs}
\usepackage{tabularx}
\newcolumntype{Y}{>{\raggedleft\arraybackslash}X}
\newcolumntype{C}{>{\centering\arraybackslash}X}
\newcommand{\ra}[1]{\renewcommand{\arraystretch}{#1}}

% Redefining the abstract to have width=\textwidth
\renewenvironment{abstract}
  {\noindent
   \begin{minipage}[t]{\textwidth}
   \setlength{\parindent}{2em}
   \renewcommand{\baselinestretch}{1.0}
   \footnotesize}
  {\end{minipage}}
  
  
%=====================================================================
% Some stuff for captioning figures
\usepackage[%
format=plain,
labelformat=simple,
labelsep=newline,
justification=centerfirst,
labelfont={footnotesize,sc},
textfont={footnotesize}
]{caption}
\setlength{\abovecaptionskip}{8pt plus 0pt minus 2pt}
%\setlength{\abovecaptionskip}{0pt}
\setlength{\belowcaptionskip}{-2pt}

\newcommand{\mycaption}[2]{%
	\caption[#1]{%
		\begin{minipage}[]{\textwidth}
		\setlength{\parindent}{1em}
		\vspace{.25cm}
		\begin{center}
		#1
		\end{center}
		\vspace{-.25cm}
		
		#2
		\end{minipage}
	}
}

\renewcommand{\thetable}{\Roman{table}}
\renewcommand{\thefigure}{\Roman{figure}}
\renewcommand{\thesubtable}{\Alph{subtable}}
\renewcommand{\thesubfigure}{\Alph{subfigure}}

\usepackage[dotinlabels]{titletoc}

\dottedcontents{section}% section
               [3em]% how far in (from margin) do you want the title to start?
               {\addvspace{.5em}\footnotesize\sc}% before-code
               {3em}% how far back (from the title words) do you want the number to appear?
               {0.75em}% how much elbow-room do the dots require

\dottedcontents{subsection}%
               [6em]%
               {\addvspace{.25em}\footnotesize\itshape}%
               {3em}%
               {0.75em}

\dottedcontents{subsubsection}%
               [9em]%
               {\addvspace{.25em}\footnotesize\itshape}%
               {3em}%
               {0.75em}

\setcounter{tocdepth}{3}

%=====================================================================
% Useful macros

\renewcommand{\L}{\mathcal{L}}
\newcommand{\U}{\mathcal{U}}

\newcommand{\R}{\mathbb{R}}
\newcommand{\N}{\mathbb{N}}
\newcommand{\Z}{\mathbb{Z}}

\newcommand{\E}{E}
\newcommand{\giv}{\;|\;}
\newcommand{\eps}{\varepsilon}

%\usepackage{amsmath}
\DeclareMathOperator*{\argmax}{arg\,max}
\DeclareMathOperator*{\argmin}{arg\,min}
\DeclareMathOperator{\sign}{sgn}

\renewcommand{\dot}[1]{\overset{\textrm{\large{.}}}{#1}}

%=====================================================================
% Latin macros

\newcommand{\ie}{\emph{i.e.},~}
\newcommand{\Ie}{\emph{I.e.},~}
\newcommand{\eg}{\emph{e.g.}~}
\newcommand{\cetpar}{\emph{cet. par.},~}
\newcommand{\parpas}{\emph{pari passu}~}
\newcommand{\qua}{\emph{quaesitum}~}

%=====================================================================
% Shortcuts for brackets/parentheses of different sizes

\newcommand{\bigb}[1]{\big[ #1 \big]}
\newcommand{\Bigb}[1]{\Big[ #1 \Big]}
\newcommand{\biggb}[1]{\bigg[ #1 \bigg]}
\newcommand{\Biggb}[1]{\Bigg[ #1 \Bigg]}

\newcommand{\bigp}[1]{\big( #1 \big)}
\newcommand{\Bigp}[1]{\Big( #1 \Big)}
\newcommand{\biggp}[1]{\bigg( #1 \bigg)}
\newcommand{\Biggp}[1]{\Bigg( #1 \Bigg)}

\newcommand{\bigc}[1]{\big\{ #1 \big\}}
\newcommand{\Bigc}[1]{\Big\{ #1 \Big\}}
\newcommand{\biggc}[1]{\bigg\{ #1 \bigg\}}
\newcommand{\Biggc}[1]{\Bigg\{ #1 \Bigg\}}


%=====================================================================
% Not sure how this works, but it pushes stuff to the right if you need that within align.

\makeatletter
\newcommand{\pushright}[1]{\ifmeasuring@#1\else\omit\hfill$\displaystyle#1$\fi\ignorespaces}
\newcommand{\pushleft}[1]{\ifmeasuring@#1\else\omit$\displaystyle#1$\hfill\fi\ignorespaces}
\makeatother



\usepackage{xcolor}

% lighter color palette
\definecolor{OperaBlue}{HTML}{87A7B7}
\definecolor{OperaMauve}{HTML}{B784A7}
\definecolor{OperaGreen}{HTML}{A7B784}

% preferred triadic color palette
\definecolor{BlueTaupe}{HTML}{5F6D91}
\definecolor{MauveTaupe}{HTML}{915F6D}
\definecolor{GreenTaupe}{HTML}{6D915F}

% darker color palette
\definecolor{OldBlue}{HTML}{314767}
\definecolor{OldMauve}{HTML}{673147}
\definecolor{OldGreen}{HTML}{476731}


% Cambridge color scheme
\definecolor{CamRed}{HTML}{D50032}
\definecolor{CamBlue}{HTML}{0072CE}
\definecolor{CamOrange}{HTML}{E87722}
\definecolor{CamGreen}{HTML}{64A70B}
\definecolor{CamPurple}{HTML}{93328E}
\definecolor{CamTeal}{HTML}{00B0B9}


\usepackage{hyperref}
\hypersetup{
  colorlinks = true, %Colours links instead of ugly boxes
  urlcolor   = blue, %Colour for external hyperlinks
  linkcolor  = blue, %Colour of internal links
  citecolor  = blue  %Colour of citations
}
\usepackage[nameinlink,noabbrev]{cleveref}




